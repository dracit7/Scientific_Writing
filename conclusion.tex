\section{Conclusion}\label{sec:conclusion}

In this notes, we illustrate causes and solutions of the cache coherency problem, one of the most representative problem in the cache maintenance area.
This problem arises in multiprocessor systems because local cache writes can break the consistency among processor caches, causing erroneous and outdated data to be read.
Existing solutions tend to use cache coherency protocols implemented by snooping or directory-based techniques to achieve write propagation and transaction serialization, ensuring the correctness of multiprocessor systems.

Interestingly, as a problem extensively studied since 1980s, the cache coherency problem is still open.
New architectures like manycores and distributed shared memory keep bringing new challenges to this area, seeking for innovative solutions.
Future research interests in this area mainly concentrate on the following directions:

\begin{itemize}
  \item Design or utilize new hardware to implement more efficient cache coherency protocols in specific scenarios (e.g., read-time systems~\cite{kaushik:asatatwclahpupcc,wu:ahpfepccpfrtm}, distributed shared memory~\cite{franques:widir}).
  \item Cache coherency schemes on disaggregated architectures like CPU-GPU~\cite{yudha:asccsficgs}, Multi-GPU~\cite{ren:hmg}.
  \item Enforcing cache coherency protocols to defend possible cyber attacks~\cite{gupta:schfmccacda,kao:ghostwriter}.
\end{itemize}

