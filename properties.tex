\section{Key Properties}\label{sec:prop}

Formally speaking, all cache coherency protocols have two key properties to ensure the consistency among processor caches:

\begin{itemize}
  \item \textbf{Write propagation}: changes to the cache block in any cache must be propagated to its copies in other processor caches.
  \item \textbf{Transaction Serialization}: accesses to a single memory location must be seen by all processors in the same order.
\end{itemize}

Write propagation must be guaranteed by protocol design, but the specific approach is not delimited.
In the three protocols introduced in~\autoref{sec:algo}, write propagation is supported by write-invalidate, which is invalidating other caches to force them reading changes from the main memory before any further transactions.
However, there are more approaches to implement write propagation, like write-update, which is informing other caches about changes initiatively instead of waiting them to trigger read or write misses.

Transaction serialization, different from write propagation, depends more on specific implementation instead of algorithmic design.
Snooping techniques leverages the shared bus to pass messages among caches, which has inherent atomicity and ordering guarantee.
In opposite, directory-based solutions have to rely on locking the directory to ensure transaction serialization, which restricts their performance in less-core scenarios.