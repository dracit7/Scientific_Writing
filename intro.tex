\section{Introduction}\label{sec:intro}

Cache, as an important component in software or hardware systems, aims to store temporary data for accelerating further requests.
Being widely applied in varied computer systems, cache typically provides lower latency and higher throughput, benefitting the systems' overall performance.
For instance, processor cache accelerates memory accesses by storing recently visited memory unit's value in faster storage media;
web cache records received large files, which are typically multimedia files in web pages, in local storage to reduce network traffic and response time.

Regardless of the cache type, maintaining the cache is always an essential task for ensuring its correctness, efficiency, and security.
Not well maintained, outdated cache might provide wrong value to requester, while complicated cache maintenance can bring significant performance degradation to computer systems~\cite{katz:iaccp}.
Moreover, unprotected cache can be leveraged to launch powerful side-channel attacks, severely threatening the system's security and reliability~\cite{fangfei:sidechannel}.
To make full use of cache while avoiding performance and security implications, researchers have proposed numerous cache maintenance techniques in the last decades.

Considering cache maintenance algorithms' complexity, variaty, and quantity, we focus on algorithms to maintain cache coherency in this notes.
We start by introduce the cache coherency problem to motivate the designation of cache coherency protocols (\autoref{sec:problem}), then elaborate on the classic MSI protocol, as well as its improved variants, to illustrate how cache maintenance algorithms are designed (\autoref{sec:algo}).
We also talk about cache coherency protocols' directory-based implementation that are applied in large-scale real-world systems because of their great scalability (\autoref{sec:practical}), key properties these protocols must guarantee (\autoref{sec:prop}), and case studies that emphasize cache coherency algorithms' importance in multiprocessor systems (\autoref{sec:case}).
We end with a high-level conclusion that summarizes existing solutions of the cache coherency problem, open problems, and future research directions (\autoref{sec:conclusion}).